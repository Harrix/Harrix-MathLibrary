\newpage

\section{Описание}

\textbf{Сайт}: \href{https://github.com/Harrix/HarrixMathLibrary}{https://github.com/Harrix/HarrixMathLibrary}.

\textbf{Что это такое?} Сборник различных математических функций и шаблонов с открытым кодом на языке C++. Упор делается на алгоритмы искусственного интеллекта. Используется только C++.

\textbf{Что из себя это представляет?} Фактически это .cpp и .h файл с исходниками функций и шаблонов, который можно прикрепить к любому проекту на C++. В качестве подключаемых модулей используется только: stdlib.h, time.h, math.h. Также используются файлы сторонней библиотеки в виде файлов mtrand.cpp и mtrand.h, для генерации псевдослучайных чисел авторства Takuji Nishimura, Makoto Matsumoto, Jasper Bedaux.

\textbf{Сколько?} На данный момент опубликовано функций: \textbf{