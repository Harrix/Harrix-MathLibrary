\textbf{Входные параметры:} 
 
VHML\_ResultVector --- сюда сохраняется результат (количество элементов, как и в других векторах VHML\_N);
 
X --- выборка: значения входов;
 
Y --- выборка: соответствующие значения выходов;
 
VHML\_N --- размер выборки;
 
C --- коэффициент размытости;
 
V --- тип ядра
 
 \begin{itemize}
 \item  0 --- прямоугольное (не рекомендуется);
 \item  1 --- треугольное;
 \item  2 --- параболическое (считается оптимальным);
 \item  3 --- экспоненциальное;
 \end{itemize}

\textbf{Возвращаемое значение:}
 
Отсутствует.

\textbf{Формула:}
\begin{eqnarray*}
\overline{dY}_j =\dfrac{\sum_{i=1}^{N}\overline{Y}_i{\Phi}'\left( \frac{\overline{X}_j-\overline{X}_i}{c}\right) \sum_{i=1}^{N}\Phi\left( \frac{\overline{X}_j-\overline{X}_i}{c}\right)-\sum_{i=1}^{N}{\Phi}'\left( \frac{\overline{X}_j-\overline{X}_i}{c}\right) \sum_{i=1}^{N}\overline{Y}_i\Phi\left( \frac{\overline{X}_j-\overline{X}_i}{c}\right)}{c\left( \sum_{i=1}^{N}\Phi\left( \frac{\overline{X}_j-\overline{X}_i}{c}\right)\right)^2 }, j=\overline{1,N}.
\end{eqnarray*}