\newpage
\section{О случайных числах в библиотеке HarrixMathLibrary}\label{section_random}

\textbf{Генератор случайных чисел (ГСЧ)} --- очень важная и нужная функция в программировании. При этом необходим лишь первичный генератор --- генератор случайных вещественных чисел в интервале $\left( 0; 1\right)$ по равномерному закону распределения (везде, где пишется о генераторе случайных чисел, надо понимать, что говорится о генераторе псевдослучайных чисел). Все остальные случайные числа с другими законами распределения можно получить из равномерного.

По умолчанию в библиотеке используется генератор случайных чисел Mersenne Twister. Также есть стандартный генератор случайных чисел.


Итак, что есть в библиотеке? Есть три функции и несколько переменных. Рассмотрим функции:
\begin{itemize}
\item \textbf{HML\_SeedRandom()} --- инициализатор генератора случайных чисел. Нужно вызвать один раз за всё время запуска программы, в которой используется библиотека.
\item \textbf{HML\_RandomNumber()} --- непосредственно генератор случайных чисел.
\item \textbf{HML\_SetRandomNumberGenerator(TypeOfRandomNumberGenerator T)} --- функция позволяет переназначить генератор случайных чисел на другой.
\end{itemize}

В файле \textbf{HarrixMathLibrary.h} (после объявления констант в начале файла) есть строчки, которые объявляют эти вещи:
\begin{lstlisting}[label=random_h,caption=Объявление функций в HarrixMathLibrary.h]
enum TypeOfRandomNumberGenerator { StandardRandomNumberGenerator, MersenneTwisterRandomNumberGenerator };//тип генератора случайных чисел: стандартный или MersenneTwister:
void HML_SeedRandom(void);//Инициализатор генератора случайных чисел
double HML_RandomNumber(void);//Генерирует вещественное случайное число из интервала (0;1)
void HML_SetRandomNumberGenerator(TypeOfRandomNumberGenerator T);//Переназначить генератор случайных чисел на другой
\end{lstlisting}

\begin{lstlisting}[label=random_h_cpp,caption=Объявление переменной в HarrixMathLibrary.cpp]
//ДЛЯ ГЕНЕРАТОРОВ СЛУЧАЙНЫХ ЧИСЕЛ
unsigned int HML_Dummy;//Результат инициализации гстандартного генератора случайных чисел
TypeOfRandomNumberGenerator HML_TypeOfRandomNumberGenerator;//тип генератора случайных чисел
MTRand mt((unsigned)time(NULL));//Инициализатор генератора случайных чисел Mersenne Twister
MTRand drand;//Для генерирования случайного числа в диапозоне [0,1).
\end{lstlisting}

В случае своего желания Вы можете заменить тело функций HML\_SeedRandom() и HML\_RandomNumber() на свои собственные. Ниже представлены варианты, которые предлагаются автором.

\begin{lstlisting}[label=random_standard,caption=Стандартный вариант по умолчанию]
void HML_SeedRandom(void)
{
/*
Инициализатор генератора случайных чисел.
В данном случае используется самый простой его вариант со всеми его недостатками.
Входные параметры:
 Отсутствуют.
Возвращаемое значение:
 Отсутствуют.
*/
//StandardRandomNumberGenerator
//Инициализатор стандартного генератора случайных чисел
//В качестве начального значения для ГСЧ используем текущее время
HML_Dummy=(unsigned)time(NULL);
srand(HML_Dummy);//Стандартная инициализация
//rand();//первый вызов для контроля

//MersenneTwisterRandomNumberGenerator
//Инициализатор генератора случайных чисел Mersenne Twister
//В качестве начального значения для ГСЧ используем текущее время
//Инициализациz происходит еще при подключении данного файла

//Назначаем генератор по умолчанию как Mersenne Twister
HML_TypeOfRandomNumberGenerator = MersenneTwisterRandomNumberGenerator;
}
//---------------------------------------------------------------------------
double HML_RandomNumber(void)
{
/*
Генератор случайных чисел (ГСЧ).
Есть два варианта генератора случайных чисел, который можно переключать
функцией HML_SetRandomNumberGenerator.
Входные параметры:
 Отсутствуют.
Возвращаемое значение:
 Случайное вещественное число из интервала (0;1) или [0;1) по равномерному закону распределения.
*/
    if (HML_TypeOfRandomNumberGenerator==StandardRandomNumberGenerator)
        return (double)rand()/(RAND_MAX+1);
    if (HML_TypeOfRandomNumberGenerator==MersenneTwisterRandomNumberGenerator)
        return drand();

    return 0;
}
//---------------------------------------------------------------------------

void HML_SetRandomNumberGenerator(TypeOfRandomNumberGenerator T)
{
/*
Функция переназначает генератор случайных чисел.
Входные параметры:
 TypeOfRandomNumberGenerator - тип генератора случайных чисел:
  StandardRandomNumberGenerator - стандартный генератор случайных чисел;
  MersenneTwisterRandomNumberGenerator - генератор случайных чисел Mersenne Twister.
Возвращаемое значение:
 Отсутствует.
*/
    HML_TypeOfRandomNumberGenerator = T;
}
//---------------------------------------------------------------------------
\end{lstlisting}

Теперь разберем, как применять данные функции.

\begin{itemize}
\item \hyperref[section_install]{Подключаем} библиотеку к Вашему проекту на C++ (об этом читайте в \hyperref[section_install]{главе об установке}).
\item В начале программы \textbf{один} раз вызываем функцию HML\_SeedRandom(). Ниже приведены примеры, где обычно стоит вызывать эту функцию.

\begin{lstlisting}[label=random_console,caption=Применение HML\_SeedRandom для консольного приложения]
int main(void)
{
HML_SeedRandom();//Инициализировали генератор случайных чисел
...
} 
\end{lstlisting}

\begin{lstlisting}[label=random_cbuilder,caption=Применение HML\_SeedRandom для C++Builder]
__fastcall TForm1::TForm1(TComponent* Owner)
       : TForm(Owner)
{
HML_SeedRandom();//Инициализировали генератор случайных чисел
...
}
\end{lstlisting}

\begin{lstlisting}[label=random_qt,caption=Применение HML\_SeedRandom для Qt]
MainWindow::MainWindow(QWidget *parent) :
    QMainWindow(parent),
    ui(new Ui::MainWindow)
{
    ui->setupUi(this);
    HML_SeedRandom();//Инициализировали генератор случайных чисел
...
}
\end{lstlisting}

\item Теперь в любом месте программы мы можем получить случайное число из интервала $ \left(0; 1\right]  $. Например:

\begin{lstlisting}[label=random_use,caption=Применение ГСЧ]
double x;
x=HML_RandomNumber();
\end{lstlisting}

Результат вызова функции, например: $ x = 0,420933187007904 $.

\end{itemize}

А с помощью функции \textbf{HML\_SetRandomNumberGenerator} можно поменять генератор случайных чисел. С помощью данного вызова меняем на стандартный генератор случайных чисел.
\begin{lstlisting}[label=SetRandomNumberGenerator,caption=Меняем на стандартный генератор случайных чисел]
HML_SetRandomNumberGenerator(StandardRandomNumberGenerator);
\end{lstlisting}

А так меняем на генератор Mersenne Twister (он стоит по умолчанию).
\begin{lstlisting}[label=SetRandomNumberGenerator,caption=Меняем на генератор случайных чисел Mersenne Twister]
HML_SetRandomNumberGenerator(MersenneTwisterRandomNumberGenerator);
\end{lstlisting}

В библиотеке используется сторонняя реализация генератора случайных чисел Mersenne Twister, и эта реализация располагается в файлах mtrand.cpp и mtrand.h.

Вы можете заменить код функций (HML\_SeedRandom, HML\_RandomNumber) на свой генератор случайных чисел в интервале $\left( 0; 1\right)$. При этом работоспособность библиотеки не нарушится.