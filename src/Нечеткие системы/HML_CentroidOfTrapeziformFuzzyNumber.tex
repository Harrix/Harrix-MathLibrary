\textbf{Входные параметры:}
 
a --- левая крайняя граница;
 
b --- начало устойчивой области;
 
с --- конец устойчивой области;
 
d --- правая крайняя граница.

\textbf{Возвращаемое значение:}
 
 Центр тяжести трапециевидного числа.

\textbf{Формула:}
\begin{equation*}
m_c=\dfrac{\int_{a}^{d} x\cdot f(x) dx}{\int_{a}^{d} f(x) dx}  =\dfrac{1}{3}\cdot\dfrac{a^2+b^2-c^2-d^2+b\cdot a-c\cdot d}{a+b-c-d}.
\end{equation*}

В случае, когда $a+b-c-d=0$ считаем, что центр масс находится посередине $(a;d)$. Должно соблюдаться, что $a\geq b \geq c \geq d$ (это на совести пользователя).