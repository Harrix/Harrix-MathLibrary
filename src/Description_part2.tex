}.

В подсчете участвуют только функции из файлов HarrixMathLibrary.cpp и HarrixMathLibrary.h. Функции из сторонней библиотеки mtrand.cpp, в которой реализован генератор случайных псевдослучайных чисел, не учитываются.

\textbf{На какие алгоритмы делается упор?} Генетические алгоритмы, алгоритмы оптимизации первого порядка и другие системы искусственного интеллекта.

\textbf{По какой лицензии выпускается?} Библиотека распространяется по лицензии Apache License, Version 2.0.

\textbf{Как найти автора?} С автором можно связаться по адресу \href {mailto:sergienkoanton@mail.ru} {sergienkoanton@mail.ru} или  \href {http://vk.com/harrix} {http://vk.com/harrix}. Сайт автора, где публикуются последние новости: \href {http://blog.harrix.org} {http://blog.harrix.org}, а проекты располагаются по адресу \href {http://harrix.org} {http://harrix.org}.

~\\

\textbf{Ваши действия:}

\begin{itemize}
\item \hyperref[section_install]{Как установить} и пользоваться библиотекой.
\item \hyperref[section_listfunctions]{Посмотреть} все функции библиотеки. Все функции рассортированы по категориям.
\item \hyperref[section_random]{Читать} о случайных числах в библиотеке.
\item \hyperref[section_addnew]{Как добавить} свои новые функции в библиотеку.
\end{itemize}