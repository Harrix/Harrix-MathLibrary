\textbf{Входные параметры:}
 
Parent1 --- первый родитель;
 
Parent2 --- второй родитель;
 
VHML\_ResultVector --- потомок;

 
VHML\_N --- размер векторов Parent1, Parent2 и VHML\_ResultVector.

\textbf{Возвращаемое значение:}

 Отсутствует.
 
\textbf{ Примечание:}

 Потомок только один.
 
Данный оператор скрещивания используется для вещественных векторов.

Пусть имеется два родителя (родительские хромосомы) $ \overline{Parent}^1 $ и $ \overline{Parent}^2$ длины $n$. Гены потомка генерируются как случайное вещественное число в границах соответствующих генов родителей:
\begin{align}
\label{SetOfOperatorsAlgorithms:eq:FlatCrossoverForReal}
&Crossover \left( \overline{Parent}^1, \overline{Parent}^2, DataOfCros\right)= \overline{Offspring}, \\
& \overline{Offspring}_i=random\left(\min\left(\overline{Parent}^1_i, \overline{Parent}^2_i \right),\max\left(\overline{Parent}^1_i, \overline{Parent}^2_i \right)  \right);\nonumber\\
&\overline{Offspring}\in X.\nonumber
\end{align}

$ DataOfCros $ не содержит каких-либо параметров относительно данного типа скрещивания.

\begin{equation}
DataOfCros=\left( \begin{array}{c} TypeOfCros \\ w \end{array} \right).
\end{equation}