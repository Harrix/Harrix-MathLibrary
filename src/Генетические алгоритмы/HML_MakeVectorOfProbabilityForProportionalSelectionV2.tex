\textbf{Входные параметры:}
 
Fitness --- массив пригодностей (можно подавать не массив пригодностей, а массив значений целевой функции, но только для задач безусловной оптимизации);
 
VHML\_ResultVector --- вектор вероятностей выбора индивидов из популяции, который мы и формируем;
 
VHML\_N --- размер массива пригодностей.

\textbf{Возвращаемое значение:} 

Отсутствует.

\textbf{О функции:}

 Это служебная функция для использования функции пропорциональной селекции HML\_SelectionProportionalV2.
Формирование вектора происходит согласно правилам пропорциональной селекции из ГА.
Работает в связке с функцией HML\_SelectionProportionalV2. Оператор селекции работает с массивом пригодностей индивидов, но непосредственно пропорциональная селекция выбирает индивида исходя из вероятностей выбора индивидов. Каждый раз для выбора индивида создавать массив вероятностей затратно, поэтому для каждой популяции на каждом поколении вначале вызывается функция HML\_MakeVectorProbabilityForSelectionProportionalV2 для генерации вектора вероятностей выбора индивида, а затем этот массив и подставляется в пропорциональную селекцию.

\textbf{Примечание:}

 Под массивом пригодностей понимается специально преобразованный массив значений целевой функции. Процесс подробно описан в стандарте генетического алгоритма. Смотреть здесь. Но это если Вы используете в алгоритмах оптимизации подобных генетическому. А так, если будете использовать, то учитывайте, что массив пригодностей --- это массив вещественных чисел из отрезка $[0;1]$.
 
 \textbf{Примечание:} 

Не используйте эту функцию над векторами целых типов int, long. Вектор обнулится кроме одно какого-нибудь элемента, так как нормировка вектора предполагает числа из интервала $[0;1]$.