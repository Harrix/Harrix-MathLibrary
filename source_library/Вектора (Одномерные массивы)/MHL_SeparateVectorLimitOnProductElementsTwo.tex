Задача возникла при построении таблицы, в которой надо показать все возможные комбинации (поэтому и произведение учитывается), когда из каждой группы берем по одному элементу, а по горизонтали слишком много элементов не расставим.

\textbf{Входные параметры:}  
 
VMHL\_Vector --- указатель на вектор, в котором хранится количество элементов в каждой группе (все должн быть положительны);
 
Order --- массив, в котором сохраняется новый порядок элементов. То есть это строка перестановка, где значение элемента говорит, что на этой позиции должен находится соответствующая группа из VMHL\_Vector.
 
Limit --- какое максимальное произведение элементов должно быть в первой группе.
 
VMHL\_N --- размер массива VMHL\_Vector и Order.
 
\textbf{Возвращаемое значение:}

Количество первых элементов в Order, которые относятся к первой группировке групп. Если это невозможно, то возвращается -1 (в случае, если минимальный элемент).
