\textbf{Входные параметры:}

VMHL\_Vector --- выходной вектор, в который записывется результат;
 
I --- номер в массиве в полном переборе, начиная с нуля (от 0 и до произведения всех элементов массива HowMuchInElements - 1);
 
HowMuchInElements --- сколько значений может принимать элемент в векторе. То есть элемент может быть 0 и HowMuchInElements[i]-1;
 
VMHL\_N --- количество элементов в массиве.

\textbf{Возвращаемое значение:}
 
Отсутствует.

\textbf{Примечание:}
 
Где может быть использована эта функция? Допустим, у вас есть десять вложеннных циклов, в которых меняется какой-то параметр. Плюс вложенность циклов может еще и варьироваться. И с помощью данной функции все эти вложенные циклы заменяются на один, а значения счетчиков вычисляются с помощью этой функции.