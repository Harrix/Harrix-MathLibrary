\textbf{Входные параметры:}
 x --- точка, в которой считается производная;
 
 h --- малое приращение x;
 
 Function --- функция, производная которой ищется.

\textbf{Возвращаемое значение:}
 
 Значение производной в точке.
 
 \textbf{Примечание:}
 
 При $h\leq0$ возвращается $0$.

\textbf{Формула:}
\begin{eqnarray*}
f'\left( x\right) \approx \dfrac{f\left( x+h\right)-f\left( x\right) }{h}
\end{eqnarray*}

Будем использовать в примере использования дополнительную функцию.

\begin{lstlisting}[caption=Дополнительная функция]
double Func3(double x)
{
return x*x;
}
//---------------------------------------------------------------------------
\end{lstlisting}