\textbf{Входные параметры:}

\textbf{i} - номер подзадачи (начиная с нуля).

В переопределяемой функции также есть параметр:
  
Type --- обозначение тестовой функции, которую вызываем. 

Смотреть виды в переменных перечисляемого типа в начале HarrixMathLibrary.h файла: TestFunction\_Ackley, TestFunction\_ParaboloidOfRevolution, TestFunction\_Rastrigin и др. Они совпадают с названиями одноименных тестовых функций, но без приставки \textbf{MHL\_}.

\textbf{Возвращаемое значение:}
 
Размерность тестовой задачи для тестовой функции вещественной оптимизации.

Итак, для обычного использования (без параметра Type) нужно вызвать функцию MHL\_DefineTestFunction. Иначе использовать переопределенную функцию и самому указать тип тестовой функции.