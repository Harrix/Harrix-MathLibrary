\textbf{Входные параметры:}

x --- указатель на исходный массив;

VMHL\_N --- размер массива x.

В переопределяемой функции также есть параметр:
  
Type --- обозначение тестовой функции, которую вызываем.
Смотреть виды в переменных перечисляемого типа в начале MathHarrixLibrary.h файла: TestFunction\_Ackley, TestFunction\_ParaboloidOfRevolution, TestFunction\_Rastrigin и др. Они совпадают с названиями одноименных тестовых функций, но без приставки \textbf{MHL\_}.

\textbf{Возвращаемое значение:}
 
Значение тестовой функции в точке x.

Итак, для обычного использования (без параметра Type) нужно вызвать функцию MHL\_DefineTestFunction. Иначе использовать переопределенную функцию и самому указать тип тестовой функции.

Все функции так высчитываются, чтобы алгоритм решал задачу поиска максимального значения целевой функции, поэтому тестовые функции на минимум умножаются на $-1$.