\textbf{Входные параметры:}
 
Parameters --- Вектор параметров генетического алгоритма. Каждый элемент обозначает свой параметр:
 
 \begin{itemize}
 \item   [0] --- длина вещественной хромосомы (определяется задачей оптимизации, что мы решаем);
  \item   [1] --- число вычислений целевой функции (CountOfFitness);
  \item    [2] --- тип селекции (TypeOfSel):
 \begin{itemize}
       \item 0 --- ProportionalSelection (Пропорциональная селекция);
 
       \item 1 --- RankSelection (Ранговая селекция);
 
       \item 2 --- TournamentSelection (Турнирная селекция).
	    \end{itemize}
 
 \item [3] --- тип скрещивания (TypeOfCros):
  \begin{itemize}
       \item 0 --- SinglepointCrossover (Одноточечное скрещивание);
 
       \item 1 --- TwopointCrossover (Двуточечное скрещивание);
 
       \item 2 --- UniformCrossover (Равномерное скрещивание).
	    \end{itemize}
 
 \item [4] --- тип мутации (TypeOfMutation):
  \begin{itemize}
       \item 0 --- Weak (Слабая мутация);
 
       \item 1 --- Average (Средняя мутация);
 
       \item 2 --- Strong (Сильная мутация).
	    \end{itemize}
 
 \item [5] --- тип формирования нового поколения (TypeOfForm):
  \begin{itemize}
       \item 0 --- OnlyOffspringGenerationForming (Только потомки);
 
       \item 1 --- OnlyOffspringWithBestGenerationForming (Только потомки и копия лучшего индивида).
	    \end{itemize}
 \item [6] --- тип преобразования задачи вещественной оптимизации в задачу бинарной оптимизации (TypOfConverting);
   \begin{itemize}
        \item 0 --- IntConverting (Стандартное представление целого числа –-- номер узла в сетке дискретизации);
        \item 1 --- GrayСodeConverting (Стандартный рефлексивный Грей-код).
			    \end{itemize}
				
 \item [7] --- <<доля>> (Proportion) числа поколений от общего числа вычислений целевой функции. Определяется как возведение числа вычислений целевой функции в степень Proportion. Может принимать значения в интервале $[0;1]$. Число поколений = int(CountOfFitness$\textasciicircum$Proportion). Размер популяции = int(CountOfFitness/Число поколений). При Proportion=0.5 получим обычный стандартный генетический алгоритм. Чем меньше Proportion, тем меньше число поколений будет. Желательно, чтобы принимались следующие значения:
  \begin{itemize}
       \item 0;
	   \item 0.1;
	   \item 0.2;
	   \item 0.3;
	   \item 0.4;
	   \item 0.5;
	   \item 0.6;
	   \item 0.7;
	   \item 0.8;
	   \item 0.9; 
       \item 1.
	    \end{itemize}
 \end{itemize}
 
 NumberOfParts --- указатель на массив: на сколько частей делить каждую вещественную координату при дискретизации (размерность Parameters[0]);
 
  Желательно брать по формуле $NumberOfParts[i]=2^k-1$, где $k$ --- натуральное число, например, 12.
  
 Left --- массив левых границ изменения каждой вещественной координаты (размерность Parameters[0]);
 
 Right --- массив правых границ изменения каждой вещественной координаты (размерность Parameters[0]);
 
 FitnessFunction --- указатель на целевую функцию (если решается задача условной оптимизации, то учет ограничений должен быть включен в эту функцию);
 
 VMHL\_ResultVector --- найденное решение (вещественный вектор);
 
 VMHL\_Result --- значение целевой функции в точке, определенной вектором VMHL\_ResultVector.

\textbf{Возвращаемое значение:} 

 1 --- завершил работу без ошибок. Всё хорошо.
 
 0 --- возникли при работе ошибки. Скорее всего в этом случае в VMHL\_ResultVector и в VMHL\_Result не содержится решение задачи.

\textbf{О функции:}

Отличается от стандартного генетического алгоритма, что число поколений может изменяться по описанному выше принципу.

Алгоритм вещественной оптимизации. Ищет максимум целевой функции FitnessFunction.

Решением является бинарная строка, то есть вектор, состоящий из 0 и 1.

Подробное описание алгоритма можно найти тут:

\href{https://github.com/Harrix/HarrixOptimizationAlgorithms/blob/master/\_HarrixOptimizationAlgorithms.pdf}{https://github.com/Harrix/HarrixOptimizationAlgorithms}

\textbf{Примерный настройки} (для примера Вы можете поставить такие рабочие настройки):

 Parameters[0]=50;
 
Parameters[1]=100*100;

Parameters[2]=2;

Parameters[3]=2;

Parameters[4]=1;

Parameters[5]=1;

Parameters[6]=0;

Parameters[7]=0.5;

Код целевой функции:
\begin{lstlisting}[caption=Оптимизируемая функция]
double Func2(double *x,int VMHL_N)
{
return -((x[0]-2)*(x[0]-2)+(x[1]-2)*(x[1]-2));
}
\end{lstlisting}