\newpage
\section{Как добавлять новые функции в библиотеку}\label{section_addnew}

Данная глава предназначена для тех, кто хочет добавлять в библиотеку новые функции и развивать данный продукт.

\textbf{Ваши действия:}

\begin{itemize}
\item \hyperref[step0]{Шаг 0}. Прочитать некоторую справочную информацию.
\item \hyperref[step1]{Шаг 1}. Написать и проверить свою функцию в папке \textbf{source\_demo}.
\item \hyperref[step2]{Шаг 2}. Раскидать в функцию по файлам в папке исходников \textbf{source\_library}.
\item \hyperref[step3]{Шаг 3}. Собрать библиотеку в папке \textbf{make}.
\item \hyperref[step4]{Шаг 4}. Раскидать файлы собранной библиотеки из папки \textbf{temp\_library} по папкам библиотеки и перекомпилировать некоторые программы и справки.
\end{itemize}

\textbf{Шаг 0.} \label{step0} Справочная информация.

Вначале надо сориентироваться в структуре библиотеки:
\begin{itemize}
\item \textbf{\_library} --- основная папка, в которой распологается готовая библиотека и данная справка;
\item \textbf{demo} --- папка, в которой находится программа DemoHarrixMathLibrary.exe с демонстрацией работы функций;
\item \textbf{make} --- в этой папке находится программа MakeHarrixMathLibrary.exe для собрания готовых файлов библиотеки из исходных материалов из папки source\_library. Также там находится справка по этой программе;
\item \textbf{source\_demo} --- папка с исходными кодами DemoHarrixMathLibrary.exe из папки demo;
\item \textbf{source\_library} --- папка исходных материалов библиотеки. Сами эти файлы библиотекой не являются, так как они потом собираются MakeHarrixMathLibrary.exe; 
\item \textbf{source\_make} --- папка с исходными кодами MakeHarrixMathLibrary.exe из папки make;
\item \textbf{LICENSE.txt} и \textbf{NOTICE.txt} --- файлы Apache лицензии;
\item \textbf{README.md} --- файл информации о проекте в системе GitHub.
\end{itemize}

Для полноценной работы по добавлению функций вам потребуются:
\begin{itemize}
\item программа для проверки работоспособности новых функций и компиляции DemoHarrixMathLibrary.exe (например, Qt 5.1.0 с Qt Creator 2.7.0 или любая другая версия Qt). Для проверки работоспособности библиотеки без компиляции DemoHarrixMathLibrary.exe подойдет любой другой C++ компилятор;
\item программа для компиляции *.tex документов в *.pdf для формирования справочных материалов. Автор использует для этого связку MiKTex и TeXstudio (версии MiKTex 2.9 и TeXstudio 2.5.2).
\end{itemize}

В варианте, который использует автор, в *.tex файлах справкок для отображения русских букв используется модуль \textbf{pscyr}. Об его установке можно прочитать в статье \href{http://blog.harrix.org/?p=444}{http://blog.harrix.org/?p=444}.

Далее приведены некоторые спецификации, принятые в данной библиотеке.
\begin{itemize}
\item Основу библиотеки составляют функции и шаблоны функций. Имена функций начинаются с \textbf{MHL\_}, например:
\begin{lstlisting}[label=examplename,caption=Пример названия функции]
void MHL_NormalizationVectorOne(double *VMHL_ResultVector,int VMHL_N);
\end{lstlisting}
Имена же шаблонов начинаются с \textbf{TMHL\_}, например:
\begin{lstlisting}[label=examplename2,caption=Пример названия шаблона функции]
template <class T> int TMHL_SearchFirstZero(T *VMHL_Vector, int VMHL_N);
\end{lstlisting}
Код функций в итоге будет располагаться в HarrixMathLibrary.cpp, а реализация шаблонов будет располагаться в HarrixMathLibrary.h.
\item  Количество элементов в одномерном массиве обозначается стандартной переменной  \textbf{int VMHL\_N}.
\item Количество элементов в двумерном массиве обозначается стандартными переменными  \textbf{int VMHL\_N} и \textbf{int VMHL\_M}.
\item Возвращаемое значение функций обозначается переменной \textbf{VMHL\_Result}.
\item Возвращаемый вектор (над которым производятся вычисления) обозначается указателем \textbf{*VMHL\_ResultVector}.
\item Возвращаемая матрица (над которой производятся вычисления) обозначается указателем \textbf{**VMHL\_ResultMatrix}.
\item Если функция в качестве параметра имеет одну числовую переменную, то она обозначается \textbf{VMHL\_X} или \textbf{VMHL\_X1}. Если есть однотипные переменные, то обозначаются \textbf{VMHL\_X2} или \textbf{VMHL\_Y} и так далее.
\item Если функция в качестве параметра имеет некий вектор, то он обозначается \textbf{VMHL\_Vector}.
\item Если функция в качестве параметра имеет некую матрицу, то она обозначается \textbf{VMHL\_Matrix}.
\item То есть если входные переменные не имеют какой-то особый смысл, то название переменных стандартно, но в тоже время все входные и выходные переменные могут начинаться с \textbf{VMHL\_}, чтобы различать их от внутренних, но во отличии от выходных значений это есть \textbf{не обязательное условие}.
\end{itemize}

Далее приведена последовательность действий, которую надо выполнить для добавления новой функции. Допустим мы хотим добавить функцию  \textbf{double MHL\_Func(double VMHL\_X)}.

\textbf{Шаг 1.}\label{step1} Вначале нам нужно реализовать саму функцию и проверить ее работоспособность. Если вы хотите работать не через средства, предоставляемые библиотекой, то этот шаг можно пропустить.

\begin{itemize}
\item Заходим в папку \textbf{source\_demo} и открываем проект \textbf{DemoHarrixMathLibrary.pro} в Qt Creator.
\item Компилируем проект и в папку с скомпиированным приложением (в режиме Release) из папки \textbf{demo} скопируем файлы:
\begin{itemize}
\item  папка images;
\item  index.html;
\item  jquery.js;
\item  jsxgraphcore.js;
\item  style.css.
\end {itemize}
\item Добавляем в конец файлов \textbf{HarrixMathLibrary.cpp} и \textbf{HarrixMathLibrary.h} функцию, которую хотим добавить. Например, в HarrixMathLibrary.cpp добавляем:
\begin{lstlisting}[label=examplefunction01, caption=Что добавляем в HarrixMathLibrary.cpp]
int  MHL_Func(int VMHL_X)
{
/*
Умножает число на 2.
Входные параметры:
  x - число, которое будет умножаться.
Возвращаемое значение:
 Число, умноженное на 2.
*/
  return 2*VMHL_X;
}
\end{lstlisting}
А в HarrixMathLibrary.h добавляем:
\begin{lstlisting}[label=examplefunction02, caption=Что добавляем в HarrixMathLibrary.h]
int  MHL_Func(int VMHL_X);
\end{lstlisting}
\textbf{Замечание.} В .h файл добавляем до строчки <<\textbf{\#endif // HARRIXMATHLIBRARY\_H}>>.

\textbf{Замечание.} Если вы добавляете шаблон функции, то его реализацию надо добавлять в HarrixMathLibrary.h.
\item Теперь перейдем в проекте DemoHarrixMathLibrary.pro в файл \textbf{mainwindow.cpp}.
\item Вначале этого файла идет следующий код:
\begin{lstlisting}[label=examplefunction03, caption=mainwindow.cpp]
#include "mainwindow.h"
#include "ui_mainwindow.h"
#include <QDebug>
#include <QFile>
#include <QDesktopServices>
#include <QUrl>
#include <QDir>
#include <QStandardItemModel>

#include "HarrixMathLibrary.h"

#include "HarrixQtLibrary.h"

MainWindow::MainWindow(QWidget *parent) :
    QMainWindow(parent),
    ui(new Ui::MainWindow)
{
    ui->setupUi(this);

    DS=QDir::separator();
    path=QGuiApplication::applicationDirPath()+DS;//путь к папке

    MHL_SeedRandom();//Инициализация датчика случайных чисел

    model = new QStandardItemModel(this);
    model->setObjectName(QString::fromUtf8("model"));

    QStandardItem *item;//элемент списка

    //добавление новых элементов
    item = new QStandardItem(QString("TMHL_FillVector"));
    model->appendRow(item);

	//Сюда нужно добавить код 

	...

    //соединение модели списка с конкретным списком
    ui->listView->setModel(model);

    ui->listView->setEditTriggers(QAbstractItemView::NoEditTriggers);
}
\end{lstlisting}

\item Там, где написан комментарий <<\textbf{//Сюда нужно добавить код}>> необходимо добавить две строчки:
\begin{lstlisting}[label=examplefunction04, caption=Что добавить в mainwindow.cpp]
    item = new QStandardItem(QString("[Имя вашей функции]"));
    model->appendRow(item);
\end{lstlisting}
То есть в рассматриваемом примере вы должны добавить:
\begin{lstlisting}[label=examplefunction05, caption=Что добавить в mainwindow.cpp в примере]
    item = new QStandardItem(QString("MHL_Func"));
    model->appendRow(item);
\end{lstlisting}
Добавление данного кода добавить вашу функцию в список, которые будут отображаться в программе при запуске. По сути, удобнее было бы извлекать из обычного текстового файла. Может в будущих версиях так и сделаю, но все равно вам нужно потом писать код демонстрации функции, поэтому занесение в текстовой файл не предусмотрел.
\item Далее найдем функцию \textbf{MainWindow::on\_listView\_clicked}:
\begin{lstlisting}[label=examplefunction06, caption=MainWindow::on\_listView\_clicked]
void MainWindow::on_listView_clicked(const QModelIndex &index)
{
    Html="<!DOCTYPE HTML PUBLIC \"-//W3C//DTD HTML 4.0//EN\" \"http://www.w3.org/TR/REC-html40/strict.dtd\">\n<html><head><meta name=\"qrichtext\" content=\"1\" />\n<meta http-equiv=\"Content-Type\" content=\"text/html; charset=utf-8\" />\n<style type=\"text/css\">\np, li { white-space: pre-wrap; }\n</style></head><body style=\" font-family:'MS Shell Dlg 2'; font-size:8.25pt; font-weight:400; font-style:normal;\">\n";

    QString NameFunction;//Какая функция вызывается

    //выдергиваем текст
    NameFunction=index.data(Qt::DisplayRole).toString();

	//Сюда нужно добавить код 

	...

    //Показ итогового результата
    Html+="</body></html>";
    HQt_SaveFile(Html, path+"temp.html");
    ui->webView->setUrl(QUrl::fromLocalFile(path+"temp.html"));
}
\end{lstlisting}
\item Там, где написан комментарий <<\textbf{//Сюда нужно добавить код}>> добавляете код следующего типа:
\begin{lstlisting}[label=examplefunction07, caption=Добавление демонстрации работы функции]
    if (NameFunction=="[Имя вашей функции]")
    {
	//Реализация демонстрации функции
    }
\end{lstlisting}
Вместо <<\textbf{[Имя вашей функции]}>> пишите название вашей функции, такое же, что добавляли выше. Вместо комментария <<\textbf{//Сюда нужно добавить код}>> добавьте реализацию демонстрации вашей функции. Например, для рассматриваемого примера код будет выглядеть так:
\begin{lstlisting}[label=examplefunction08, caption=Добавление демонстрации работы функции на примере]
    if (NameFunction=="MHL_Func")
    {
        int x=5;

        //Вызов функции
        int y=MHL_Func(x);

        //Используем полученный результат
        MHL_ShowNumber (x,"Первоначальное число", "x");
        MHL_ShowNumber (y,"Умноженное число", "y");
        //Первоначальное число:
        //x=5
        //Умноженное число:
        //y=10
    }
\end{lstlisting}
\item Рассмотрим немного этот код. После вызова функции идет комментарий <<\textbf{//Используем полученный результат}>>. После него надо вывести в webView нужную информацию. Для этого лучше использовать стандартные функции, список который написан ниже.
\item  После вывода функций в виде комментариев показывается тот текст, который может продемонстрироваться при вызове функции. У нас это код:
\begin{lstlisting}[label=examplefunction09, caption=Закомментированный результат работы функции]
        //Первоначальное число:
        //x=5
        //Умноженное число:
        //y=10
\end{lstlisting}
\end{itemize}

Теперь рассмотрим какие функции используются для вывода результата. Типичными объектами, над которыми выполняются действия по выводу, являются: числа, вектора, матрицы. Их мы стандартизовано и выводим, используя некоторые функции. Так как библиотека HarrixMathLibrary может использоваться на различных системах C++, а вывод информации в каждой системе может быть разным, то функции вывода строились таким образом, чтобы внешне выглядели однотипно в любой системе C++, так как в справке к функциям из библиотеки функции вывода также будут присутствовать. Итак, использование функций внешне должно быть везде одинаковым для всех систем C++.  Поэтому вы можете их переписать под свои нужды.


\begin{itemize}
\item \textbf{MHL\_NumberToText} --- функция перевода числа в строку; 
\item \textbf{MHL\_ShowNumber} --- функция вывода числа;
\item \textbf{MHL\_ShowVector} --- функция вывода вектора (одномерного массива);
\item \textbf{MHL\_ShowVectorT} --- функция вывода вектора (одномерного массива) в строку одну, то есть это транспонированный вектор;
\item \textbf{MHL\_ShowMatrix} --- функция вывода матрицы.
\item \textbf{MHL\_ShowText} --- функция вывода просто текста.
\end{itemize}

Далее функции рассмотрены подробнее.

\begin{itemize}
\item \textbf{MHL\_ShowNumber} --- функция вывода числа. 
\begin{lstlisting}[label=examplefunction13, caption=Синтаксис функции MHL\_ShowNumber]
template <class T> void MHL_ShowNumber (T VMHL_X, QString TitleX, QString NameX);
\end{lstlisting}
Входные параметры: 
\begin{itemize}   
    \item  VMHL\_X --- выводимое число;
     \item TitleX --- заголовок выводимого числа;
     \item NameX --- обозначение числа.
\end{itemize}
Пример использования функции:
\begin{lstlisting}[label=examplefunction10, caption=Пример использования MHL\_ShowNumber]
MHL_ShowNumber (x,"Первоначальное число", "x");
//Первоначальное число:
//x=5
\end{lstlisting}
И для этой функции покажем исходный код:
\begin{lstlisting}[label=examplefunction11, caption=Реализация функции MHL\_ShowNumber]
//mainwindow.cpp
template <class T> void MainWindow::MHL_ShowNumber (T VMHL_X, QString TitleX, QString NameX)
{
    /*
    Функция выводит число VMHL_X в textEdit.
    Входные параметры:
     VMHL_X - выводимое число;
     TitleX - заголовок выводимого числа;
     NameX - обозначение числа.
    Возвращаемое значение:
     Отсутствует.
    */
    QString VMHL_Result;
    VMHL_Result=THQt_ShowNumber (VMHL_X, TitleX, NameX);// из HarrixQtLibrary.h
    Html+=VMHL_Result;
}
//---------------------------------------------------------------------------

//HarrixQtLibrary.h
template <class T> QString THQt_ShowNumber (T VMHL_X, QString TitleX, QString NameX)
{
    /*
    Функция возвращает строку с выводом некоторого числа VMHL_X с HTML кодами. Для добавление в html файл.
    Входные параметры:
     VMHL_X - выводимое число;
     TitleX - заголовок выводимого числа;
     NameX - обозначение числа.
    Возвращаемое значение:
     Строка с HTML кодами с выводимым числом.
    */
    QString VMHL_Result;

    VMHL_Result="<p><b>"+TitleX+":</b><br>";

    VMHL_Result+=NameX+"=<b><font color=\"#4200ff\">"+QString::number(VMHL_X)+"</font></b></p>\n";

    return VMHL_Result;
}
//---------------------------------------------------------------------------
\end{lstlisting}
В функции MainWindow::on\_listView\_clicked() есть еще код для сохранения и вывода значения переменной Html в виде *.html файла.

В предыдущей версии библиотеки для программы демонстрации работы функций использовалась система C++Builder 6. Там эта функции реализовывалась так:
\begin{lstlisting}[label=examplefunction12, caption=Реализация функции MHL\_ShowNumber в C++Builder 6]
template <class T> void MHL_ShowNumber (T X, AnsiString A, AnsiString B)
{
Form1->Memo1->Lines->Add(A+":");
Form1->Memo1->Lines->Add(B+" = "+AnsiString(X));
Form1->Memo1->Lines->Add("");
}
//---------------------------------------------------------------------------
\end{lstlisting}
Как видим, вид функций по внешнему виду однотипен --- различается только тип строк, который используется.

\item \textbf{MHL\_NumberToText} --- выводит число в строку. 
\begin{lstlisting}[label=examplefunction16_2, caption=Синтаксис функции MHL\_NumberToText]
template <class T> QString MainWindow::MHL_NumberToText (T VMHL_X);
\end{lstlisting}
Входные параметры: 
\begin{itemize}   
     \item VMHL\_X --- выводимое число.
\end{itemize}
Пример использования функции:
\begin{lstlisting}[label=examplefunction17, caption=Пример использования MHL\_NumberToText]
MHL_ShowNumber(Deg,"Угол "+MHL_NumberToText(Rad)+" радиан","равен в градусах");
//Угол 3.14159 радиан:
//равен в градусах=180
\end{lstlisting}

\item \textbf{MHL\_ShowVector} --- функция вывода вектора (одномерного массива). 
\begin{lstlisting}[label=examplefunction14, caption=Синтаксис функции MHL\_ShowVector]
template <class T> void MHL_ShowVector (T *VMHL_Vector, int VMHL_N, QString TitleVector, QString NameVector);
\end{lstlisting}
Входные параметры: 
\begin{itemize}   
     \item Vector --- указатель на выводимый вектор;
     \item VMHL\_N --- количество элементов вектора a;
     \item TitleVector --- заголовок выводимого вектора;
     \item NameVector --- обозначение вектора.
\end{itemize}
Пример использования функции:
\begin{lstlisting}[label=examplefunction15_2, caption=Пример использования MHL\_ShowVector]
MHL_ShowVector (a,VMHL_N,"Заполненный вектор", "a");
//Заполненный вектор:
//a =	
//5
//5
//5
//5
//5
//5
//5
//5
//5
//5

\end{lstlisting}

\item \textbf{MHL\_ShowVectorT} --- функция вывода вектора (одномерного массива) в транспонированном виде, то есть в одну строку. 
\begin{lstlisting}[label=examplefunction14_2, caption=Синтаксис функции MHL\_ShowVectorT]
template <class T> void MHL_ShowVectorT (T *VMHL_Vector, int VMHL_N, QString TitleVector, QString NameVector);
\end{lstlisting}
Входные параметры: 
\begin{itemize}   
     \item Vector --- указатель на выводимый вектор;
     \item VMHL\_N --- количество элементов вектора a;
     \item TitleVector --- заголовок выводимого вектора;
     \item NameVector --- обозначение вектора.
\end{itemize}
Пример использования функции:
\begin{lstlisting}[label=examplefunction15, caption=Пример использования MHL\_ShowVectorT]
MHL_ShowVector (a,VMHL_N,"Заполненный вектор", "a");
//Заполненный вектор:
//a = 5 5 5 5 5 5 5 5 5 5
\end{lstlisting}

\item \textbf{MHL\_ShowMatrix} --- функция вывода матрицы. 
\begin{lstlisting}[label=examplefunction16, caption=Синтаксис функции MHL\_ShowMatrix]
template <class T> void MHL_ShowMatrix (T **VMHL_Matrix, int VMHL_N, int VMHL_M, QString TitleMatrix, QString NameMatrix);
\end{lstlisting}
Входные параметры: 
\begin{itemize}   
     \item VMHL\_Matrix --- указатель на выводимую матрицу;
     \item VMHL\_N --- количество строк в матрице;
     \item VMHL\_M --- количество столбцов в матрице;
     \item TitleMatrix --- заголовок выводимой матрицы;
     \item NameMatrix --- обозначение матрицы.
\end{itemize}
Пример использования функции:
\begin{lstlisting}[label=examplefunction17_2, caption=Пример использования MHL\_ShowMatrix]
        MHL_ShowMatrix (Matrix,VMHL_N,VMHL_M,"Матрица", "x");
        //Матрица:
        //x =            
        //0	1	2	3	4
        //1	2	3	4	5
        //2	3	4	5	6
        //3	4	5	6	7
        //4	5	6	7	8
        //5	6	7	8	9
        //6	7	8	9	10  
\end{lstlisting}

\item \textbf{MHL\_ShowText} --- функция вывода просто текста. 
\begin{lstlisting}[label=examplefunction14855, caption=Синтаксис функции MHL\_ShowText]
void MHL_ShowText (QString TitleX);
\end{lstlisting}
Входные параметры: 
\begin{itemize}   
     \item TitleX - непосредственно выводимая строка.
\end{itemize}
Пример использования функции:
\begin{lstlisting}[label=examplefunction14855, caption=Пример использования MHL\_ShowText]
MHL_ShowText ("Выводимый текст");
//Выводимый текст

\end{lstlisting}


\end{itemize}

Итак, мы добавили в DemoHarrixMathLibrary.pro нашу функцию и проверили ее работоспособность. 

\textbf{Шаг 2.}\label{step2} Теперь нам нужно добавить нашу функцию в исходники. Все исходные материалы располагаются в папке \textbf{source\_library}. В ней располагаются некоторые файлы, которые нам не особы интересны (подробнее в файле справке к программе MakeHarrixMathLibrary.exe в файле \textbf{make\textbackslash MakeHarrixMathLibrary\_Help.pdf}) и папки (например, \textbf{Вектора (Одномерные массивы)}). Каждая такая папка является разделом функций в библиотеке. Вам нужно выбрать папку, в которую вы будете добавлять свою функцию или создать свою собственную, если ничто не подходит по смыслу.

Каждая функция или шаблон функции в разделе (выбранной вами папке) предоставляется следующими файлами:
\begin{itemize}
\item \textbf{<File>.cpp} или \textbf{<File>.tpp} --- код функции;
\item \textbf{<File>.h} --- заголовочный файл функции;
\item \textbf{<File>.tex} --- справка по функции;
\item \textbf{<File>.desc} --- описание функции;
\item \textbf{<File>.use} --- пример использования функции;
\item \textbf{<File>\_<name>.pdf} --- множество рисунков, необходимых для справки по функции (необязательные файлы);
\item \textbf{<File>\_<name>.png} --- множество рисунков, необходимых для справки по функции (необязательные файлы);
\end{itemize}

Без файлов <File>.cpp (или <File>.tpp), <File>.h, <File>.tex, <File>.desc, <File>.use библиотека соберется, но с ошибками, то есть каждая функция должна быть представима минимум 5 файлами (могут быть дополнительно рисунки).

Считаем далее, что вы выбрали папку \textbf{<Dir>} в папке source\_library. 

\begin{itemize}
\item Создайте в папке <Dir> текстовой файл \textbf{<File>.h}, где <File> --- это имя функции, то есть в рассматриваемом примере мы должны создать файл \textbf{MHL\_Func.h}.
\item В файл <File>.h мы добавляем объявление нашей функции, например:
\begin{lstlisting}[label=examplefileh, caption=Содержимое MHL\_Func.h]
int  MHL_Func(int VMHL_X);
\end{lstlisting}
\item В файл <File>.cpp мы добавляем код нашей функции, например:
\begin{lstlisting}[label=examplefilecpp, caption=Содержимое MHL\_Func.cpp]
int  MHL_Func(int VMHL_X)
{
/*
Умножает число на 2.
Входные параметры:
  x - число, которое будет умножаться.
Возвращаемое значение:
 Число, умноженное на 2.
*/
  return 2*VMHL_X;
}
\end{lstlisting}

Если у нас не функция, а шаблон функции, то мы создаем файл <File>.tpp (обратите внимание на расширение файла), например:
\begin{lstlisting}[label=examplefiletpp, caption=Содержимое TMHL\_FillVector.tpp]
template <class T> void TMHL_FillVector(T *VMHL_ResultVector, int VMHL_N, T x)
{
/*
Функция заполняет вектор значениями, равных x.
Входные параметры:
 VMHL_ResultVector - указатель на преобразуемый массив;
 VMHL_N - количество элементов в массиве;
 x - число, которым заполняется вектор.
Возвращаемое значение:
 Отсутствует.
*/
for (int i=0;i<VMHL_N;i++) VMHL_ResultVector[i]=x;
}
\end{lstlisting}

\item В файл <File>.desc мы добавляем описание нашей функции, например:
\begin{lstlisting}[label=examplefiledesc, caption=Содержимое MHL\_Func.desc]
Умножает число на 2.
\end{lstlisting}

\item В файл <File>.tex мы добавляем справку к нашей функции в виде куска tex кода, например:
\begin{lstlisting}[label=examplefiletex, caption=Содержимое MHL\_Func.tex]
\textbf{Входные параметры:}

 x --- входной параметр.

\textbf{Возвращаемое значение:}
Число умноженное на 2.
\end{lstlisting}

\item В файл <File>.use мы добавляем код примера использования функции, например:
\begin{lstlisting}[label=examplefileuse, caption=Содержимое MHL\_Func.use]
int x=5;

//Вызов функции
int y=MHL_Func(x);

//Используем полученный результат
MHL_ShowNumber (x,"Первоначальное число", "x");
MHL_ShowNumber (y,"Умноженное число", "y");
//Первоначальное число:
//x=5
//Умноженное число:
//y=10
\end{lstlisting}

\item Если хотите использовать рисунки в tex справке к функции, то в папку <Dir> скопируйте рисунки вида  \textbf{<File>\_<name>.pdf} и \textbf{<File>\_<name>.png}

\item Если мы используем дополнительную переменную перечисляемого типа, то добавляем ее в файл \textbf{Enum.h} в папке \textbf{source\_library}.

\item Если мы хотим использовать глобальную константу, то добавляем ее в файл \textbf{Const.h} в папке \textbf{source\_library}.

\item Если мы хотим использовать глобальную переменную, то добавляем ее в файл \textbf{AdditionalVariables.cpp} в папке \textbf{source\_library}.

\end{itemize}

\textbf{Замечание.} Если вы хотите переопределить функцию какую-нибудь, то вы добавляете переопределенные функции, их объявления в уже существующие файлы, а не создаете новые.

\textbf{Замечание.} Класс и его методы нужно оформлять в одном файле *.cpp, *.h и др., а не разбивать на несколько и прописывать каждый метод в отдельном.

Итак, мы добавили в папку source\_library нашу функцию. Теперь нужно перестроить библиотеку и провести замену файлов.

\textbf{Шаг 3.}\label{step3} Сборка библиотеки. Перейдем в папку \textbf{make} в корне файлов библиотеки. В ней есть программа MakeHarrixMathLibrary.exe и справка к ней MakeHarrixMathLibrary\_Help.pdf. 

\begin{itemize}
\item Включим программу \textbf{MakeHarrixMathLibrary.exe}.
\item Нажмем кнопку \textbf{Собрать библиотеку}.
\item В окне программы будет отчет об собрании библиотеки, например:
\begin{lstlisting}[label=examplereport, caption=Пример отчета о сборке библиотеки]
Начало формирования файлов библиотеки...
Загрузили файл Header.cpp
Загрузили файл AdditionalVariables.cpp
Загрузили файл Random.cpp
Загрузили файл Const.h
Загрузили файл Random.cpp
Загрузили файл Enum.h
Загрузили файл Install.tex
Загрузили файл Random.tex
Загрузили файл Addnew.tex

Было найдено 1 папок - разделов библиотеки

Рассматриваем папку Вектора (Одномерные массивы)
Было найдено 15 файлов в папке

Загрузили файл FuncF.cpp
Загрузили файл FuncF.desc
Загрузили файл FuncF.h
Загрузили файл FuncF.tex
Загрузили файл FuncF.use
Загрузили файл MHL_Func.cpp
Загрузили файл MHL_Func.desc
Загрузили файл MHL_Func.h
Загрузили файл MHL_Func.tex
Загрузили файл MHL_Func.use
Загрузили файл TMHL_FillVector.desc
Загрузили файл TMHL_FillVector.h
Загрузили файл TMHL_FillVector.tex
Загрузили файл TMHL_FillVector.tpp
Загрузили файл TMHL_FillVector.use
Из 15 файлов нужными нам оказалось 15 файлов в папке

Загрузили файл Description_part2.tex
Загрузили файл Description_part1.tex
Загрузили файл Title.tex

Сохранили файл HarrixMathLibrary.cpp
Сохранили файл HarrixMathLibrary.h
Сохранили файл HarrixMathLibrary_Help.tex

Скопировали файл names.tex
Скопировали файл packages.tex
Скопировали файл styles.tex

Ошибки не были зафиксированы.
Конец формирования файлов библиотеки.
Потребовалось времени: 1 мин. 9 сек. 550 миллисек.
\end{lstlisting}

Если ошибок нет, то все прошло нормально.
\item Также нам будет продемонстрирована папка \textbf{temp\_library} с сформированными файлами библиотеки.
\end{itemize}

Итак, мы собрали файлы библиотеки.

\textbf{Шаг 4.}\label{step4} Разберем файлы из папки \textbf{temp\_library}.

\begin{itemize}
\item Скопируем файлы \textbf{HarrixMathLibrary.cpp} и \textbf{HarrixMathLibrary.h} в папку \textbf{\_library}.

\item Откройте файл \textbf{HarrixMathLibrary\_Help.tex } в \LaTeX \ программе (автор использует TeXstudio) и скомпилируйте его.

В итоге в папке temp\_library появится файл \textbf{HarrixMathLibrary\_Help.pdf}. Скопируйте этот файл в папку \textbf{\_library}.

\item Теперь разберемся с программой для демонстрации. Как мы помним, в ней в самом начале мы проверяли свою функцию. 
\begin{itemize}
\item Скопируем файлы \textbf{HarrixMathLibrary.cpp} и \textbf{HarrixMathLibrary.h} в папку \textbf{source\_demo}.
\item  Откройте \textbf{DemoHarrixMathLibrary.pro} из папки source\_demo в Qt Creator и скомпилирйте приложение (в режиме Release).
\item Найдите папку, в которую скомпилировался проект. Это может быть папка проектов Qt, или папка появится в корневой папке библиотеки HarrixMathLibrary.
\item Скопируйте файл \textbf{DemoHarrixMathLibrary.exe} в папку \textbf{demo}.
\end{itemize}
\item Удалим папку \textbf{temp\_library} после всех наших действий.
\item  Если папка с скомпилированным файлом DemoHarrixMathLibrary.exe появилась в корневой папке библиотеки, то удалите ее (например, build-DemoHarrixMathLibrary-Desktop\_Qt\_5\_0\_2\_MinGW\_32bit-Release).
\item Отредактируйте на своё усмотрение файл README.md, где напишите о новых изменениях.
\item В файлах \textbf{README.md}, \textbf{Header.cpp}, \textbf{source\_library\textbackslash Title.tex} поменяйте номер версии библиотеки.
\end{itemize}

Вот, вроде и всё. Мы добавили новую функцию и обновили все файлы и папки библиотеки.