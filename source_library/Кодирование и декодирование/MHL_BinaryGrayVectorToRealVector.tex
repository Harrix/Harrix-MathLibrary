\textbf{Входные параметры:}
 
a --- бинарная строка представляющая собой Грей---код нескольких закодированных вещественных координат;
 
n --- длина бинарной строки;
 
VMHL\_ResultVector --- вещественный вектор, в который мы и записываем результат, размера n;
 
Left --- массив левых границ изменения каждой вещественной координаты (размер VMHL\_N);
 
Right --- массив правых границ изменения каждой вещественной координаты (размер VMHL\_N);
 
Lengthi --- массив значений, сколько на каждую координату отводится бит в бинарной строке (размер массива Lengthi VMHL\_N);
 
VMHL\_N --- длина вещественного вектора.
 

\textbf{Возвращаемое значение:}
 
Отсутствует.
  
Для перегруженной функции
  
\textbf{Входные параметры:}
 
a --- бинарная строка представляющая собой Грей---код нескольких закодированных вещественных координат;
 
VMHL\_ResultVector --- вещественный вектор, в который мы и записываем результат;
 
TempBinaryVector --- указатель на временный массив  размера n;
 
Left --- массив левых границ изменения каждой вещественной координаты размера VMHL\_N;
 
Right --- массив правых границ изменения каждой вещественной координаты размера VMHL\_N;
 
Lengthi --- массив значений, сколько на каждую координату отводится бит в бинарной строке. Размер массива VMHL\_N;
 
VMHL\_N --- длина вещественного вектора.
 
\textbf{Возвращаемое значение:}
 
Отсутствует.

\textbf{Примечание:}
 К криптографии данная функция не имеет отношения.
