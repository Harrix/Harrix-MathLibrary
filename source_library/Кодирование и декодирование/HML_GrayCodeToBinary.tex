\textbf{Входные параметры:}
 
 a --- код Грея (массив заполнен 0 и 1);
 
 VHML\_N --- длина массива a.
 
\textbf{Возвращаемое значение:}

 Отсутствует.
 
\textbf{О функции:}

Бинарная строка не представляет собой двоичный код целого числа, а представляет код Грея. Его отличительной особенностью является то, что если два целых числа отличаются на единицу, то их коды Грея также будут отличаться только одним битом. Двоичный код не обладает данным свойством.
Существует метод по переводу кода Грея в двоичный код: старший разряд (крайний левый бит) записывается без изменения, каждый следующий символ кода Грея нужно инвертировать, если в двоичном коде перед этим была получена «1», и оставить без изменения, если в двоичном коде был получен «0». 