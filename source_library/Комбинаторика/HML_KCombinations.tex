\textbf{Входные параметры:}  
 
 k --- по сколько элементов надо брать в группу;
 
 n --- общее число элементов.

\textbf{Возвращаемое значение:}

 Число сочетаний из n по k.
 
 \textbf{Формула:}
 
 В программном коде число сочетаний находится через рекурсивную формулу. А в математике находится через формулу:
 
 \begin{equation*}
C_n^k=\dfrac{n!}{k!\left( n-k\right)! }.
\end{equation*}