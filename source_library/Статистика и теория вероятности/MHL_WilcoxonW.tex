\textbf{Входные параметры:}

 a --- первая выборка;
 
b --- вторая выборка;
 
VMHL\_N1 --- размер первой выборки;
 
VMHL\_N2 --- размер второй выборки;
 
Q --- уровень значимости. Может принимать значения:
	
\begin{itemize}
\item 0.002;
\item 0.01; 
\item 0.02; 
\item 0.05; 
\item 0.1; 
\item 0.2.
\end{itemize}

\textbf{Возвращаемое значение:}

 -2 --- уровень значимости выбран неправильно (не из допустимого множества);
 
 -1 --- объемы выборок не позволяют провести проверку при данном уровне значимости (или они не положительные);
 
 0 --- выборки не однородны  при данном уровне значимости;
 
 1 --- выборки однородны  при данном уровне значимости;

\textbf{Примечание:}

 Если размеры выборок не из таблицы, если не правильный выбран уровень значимости, то возвратится -1.
 
 Обратите внимание, что допустимые значения значимости Q в функциях MHL\_LeftBorderOfWilcoxonWFromTable и MHL\_RightBorderOfWilcoxonWFromTable в два раза меньше. Это связано с тем, что критерий Вилкосена использует двухсторонний критерий. Поэтому уровень значимости должен быть повышен, по сравнению с табличными значениями.
	 
 Информация о критерии из  справочника <<Таблицы математической статистики>> \cite[с. 93]{book:Bolshev1983}. Описание по данной книге можно прочитать на \href {https://github.com/Harrix/Wilcoxon-W-Test} {https://github.com/Harrix/Wilcoxon-W-Test}.
