\newpage
\section{Установка}\label{section_install}

Если вы хотите только пользоваться библиотекой, то вам нужна из всего проекта только папки \textbf{\_library}, в которой располагается собранная библиотека и справка по ней, и папка \textbf{demo}, в которой находится программа с демонстрацией работы функций. Все остальные папки вам потребуются, если вы хотите добавлять новые функции.

\subsection{Общий алгоритм подключения}

\begin{itemize}
\item Скопируем себе папку \textbf{\_library} с готовой последней версией библиотеки на сайте проекта \href{https://github.com/Harrix/HarrixMathLibrary}{https://github.com/Harrix/HarrixMathLibrary}.

\item Скопируем файлы \textbf{HarrixMathLibrary.cpp}, \textbf{HarrixMathLibrary.h}, \textbf{mtrand.cpp}, \textbf{mtrand.h} в папку с Вашим проектом на C++.

\item Пропишем в проекте:
\begin{lstlisting}[label=install_01,caption=Подключение библиотеки]
#include "HarrixMathLibrary.h"
\end{lstlisting}

\item Если планируем использовать функции, использующие случайные числа (если не знаем, то тоже сделаем), то в начале программы вызовем:
\begin{lstlisting}[label=install_02,caption=Инициализация генератора случайных чисел]
MHL_SeedRandom();//Инициализировали генератор случайных чисел
\end{lstlisting}

\item Теперь библиотека готова к работе, и можем ее использовать. Например:
\begin{lstlisting}[label=install_03,caption=Пример использования]
double x;
x=MHL_RandomNumber();
double degree=MHL_DegToRad(60);
\end{lstlisting}
\end{itemize}


\subsection{Подключение к Qt на примере Qt 5.2.0}

Рассматривается на примере создания Qt Gui Application в Qt 5.2.0 for Desktop (MinGW 4.8) с использованием Qt Creator 2.8.84.

\begin{itemize}
\item Скопируем себе папку \textbf{\_library} с готовой последней версией библиотеки на сайте проекта \href{https://github.com/Harrix/HarrixMathLibrary}{https://github.com/Harrix/HarrixMathLibrary}.

\item Скопируем файлы \textbf{HarrixMathLibrary.cpp}, \textbf{HarrixMathLibrary.h}, \textbf{mtrand.cpp}, \textbf{mtrand.h} в папку с Вашим проектом на C++ там, где находится файл проекта *.pro.

\item Добавим к проекту файлы \textbf{HarrixMathLibrary.cpp}, \textbf{HarrixMathLibrary.h}, \textbf{mtrand.cpp}, \textbf{mtrand.h}. Для этого по проекту в Qt Creator щелкнем правой кнопкой и вызовем команду \textbf{Add Existings Files...}, где выберем наши файлы.

\item Пропишем в главном файле исходников проекта \textbf{mainwindow.cpp}:
\begin{lstlisting}[label=install_01_qt,caption=Подключение библиотеки]
#include "HarrixMathLibrary.h"
\end{lstlisting}

\item Если планируем использовать функции, использующие случайные числа (если не знаем, то тоже сделаем), то в начале программы в конструкторе \textbf{MainWindow::MainWindow} вызовем:
\begin{lstlisting}[label=install_02_qt,caption=Инициализация генератора случайных чисел]
MHL_SeedRandom();//Инициализировали генератор случайных чисел
\end{lstlisting}

То есть получится код:
\begin{lstlisting}[label=install_03_qt,caption=Пример файла mainwindow.cpp с подключенной библиотекой]
#include "mainwindow.h"
#include "ui_mainwindow.h"

#include "HarrixMathLibrary.h"

MainWindow::MainWindow(QWidget *parent) :
    QMainWindow(parent),
    ui(new Ui::MainWindow)
{
    ui->setupUi(this);
    MHL_SeedRandom();//Инициализировали генератор случайных чисел    
}

MainWindow::~MainWindow()
{
    delete ui;
}
\end{lstlisting}

\item Теперь библиотека готова к работе, и можем ее использовать. Например, добавим textEdit, pushButton и напишем слот кнопки:
\begin{lstlisting}[label=install_04_qt,caption=Пример использования]
void MainWindow::on_pushButton_clicked()
{
    double x;
    x=MHL_RandomNumber();
    double degree=MHL_DegToRad(60);
}
\end{lstlisting}
\end{itemize}

\subsection{Подключение к C++ Builder на примере C++ Builder 6.0}
\begin{itemize}
\item Скопируем себе папку \textbf{\_library} с готовой последней версией библиотеки на сайте проекта \href{https://github.com/Harrix/HarrixMathLibrary}{https://github.com/Harrix/HarrixMathLibrary}.

\item Скопируем файлы \textbf{HarrixMathLibrary.cpp}, \textbf{HarrixMathLibrary.h}, \textbf{mtrand.cpp}, \textbf{mtrand.h} в папку с проектом на C++.

\item Пропишем проекте в файле \textbf{.cpp} главной формы (часто это \textbf{Unit1.cpp}) строчку \textbf{\#include "HarrixMathLibrary.h"}:
\begin{lstlisting}[label=install_code_01,caption=Подключение библиотеки]
//---------------------------------------------------------------------------

#include <vcl.h>
#pragma hdrstop

#include "Unit1.h"
#include "HarrixMathLibrary.h"
//---------------------------------------------------------------------------
#pragma package(smart_init)
#pragma resource "*.dfm"
TForm1 *Form1;
...
\end{lstlisting}

\item Добавим в проект файлы \textbf{HarrixMathLibrary.cpp} и \textbf{mtrand.cpp} через команду: \textbf{Project} $\rightarrow$ \textbf{Add to Project\dots}.

\item Если планируем использовать функции, использующие случайные числа (если не знаем, то тоже сделаем), то в конструкторе главной формы инициализируем генератор случайных чисел:
\begin{lstlisting}[label=install_code_02,caption=Инициализация генератора случайных чисел]
__fastcall TForm1::TForm1(TComponent* Owner)
        : TForm(Owner)
{
MHL_SeedRandom();//Инициализировали генератор случайных чисел
...
}
//---------------------------------------------------------------------------
\end{lstlisting}

\item Теперь библиотека готова к работе, и можем ее использовать. Например, создадим кнопку Button1, текстовое поле Memo1 и в клике на Button1 пропишем:
\begin{lstlisting}[label=install_code_03,caption=Пример использования]
void __fastcall TForm1::Button1Click(TObject *Sender)
{
double x=MHL_RandomNumber();//получим случайное число
Memo1->Lines->Add("x = "+AnsiString(x));//выведем его
}
//---------------------------------------------------------------------------
\end{lstlisting}
\end{itemize}

\subsection{Подключение к C++ Builder на примере C++Builder XE4}
\begin{itemize}
\item Скопируем себе папку \textbf{\_library} с готовой последней версией библиотеки на сайте проекта \href{https://github.com/Harrix/HarrixMathLibrary}{https://github.com/Harrix/HarrixMathLibrary}.

\item Скопируем файлы \textbf{HarrixMathLibrary.cpp}, \textbf{HarrixMathLibrary.h}, \textbf{mtrand.cpp}, \textbf{mtrand.h} в папку с проектом на C++.

\item Пропишем проекте в файле \textbf{.cpp} главной формы (часто это \textbf{Unit1.cpp}) строчку \textbf{\#include "HarrixMathLibrary.h"}:
\begin{lstlisting}[label=install_code_04,caption=Подключение библиотеки]
//---------------------------------------------------------------------------

#include <vcl.h>
#pragma hdrstop

#include "Unit1.h"
#include "HarrixMathLibrary.h"
//---------------------------------------------------------------------------
#pragma package(smart_init)
#pragma resource "*.dfm"
TForm1 *Form1;
//---------------------------------------------------------------------------
...
\end{lstlisting}

\item Добавим в проект файлы \textbf{HarrixMathLibrary.cpp} и \textbf{mtrand.cpp} через команду: \textbf{Project} $\rightarrow$ \textbf{Add to Project\dots}.

\item Если планируем использовать функции, использующие случайные числа (если не знаем, то тоже сделаем), то в конструкторе главной формы инициализируем генератор случайных чисел:
\begin{lstlisting}[label=install_code_05,caption=Инициализация генератора случайных чисел]
__fastcall TForm1::TForm1(TComponent* Owner)
	: TForm(Owner)
{
MHL_SeedRandom();//Инициализировали генератор случайных чисел
...
}
//---------------------------------------------------------------------------
\end{lstlisting}

\item Теперь библиотека готова к работе, и можем ее использовать. Например, создадим кнопку Button1, текстовое поле Memo1 и в клике на Button1 пропишем:
\begin{lstlisting}[label=install_code_06,caption=Пример использования]
void __fastcall TForm1::Button1Click(TObject *Sender)
{
double x=MHL_RandomNumber();//получим случайное число
Memo1->Lines->Add("x = "+AnsiString(x));//выведем его
}
//---------------------------------------------------------------------------
\end{lstlisting}
\end{itemize}

\subsection{Подключение к Microsoft Visual Studio на примере Visual Studio 2012}

Используется СLR приложение Windows Forms Application (точнее пустой проект, к которому прикреплена форма) на Visual C++.

\begin{itemize}
\item Скопируем себе папку \textbf{\_library} с готовой последней версией библиотеки на сайте проекта \href{https://github.com/Harrix/HarrixMathLibrary}{https://github.com/Harrix/HarrixMathLibrary}.

\item Скопируем файлы \textbf{HarrixMathLibrary.cpp}, \textbf{HarrixMathLibrary.h}, \textbf{mtrand.cpp}, \textbf{mtrand.h} в папку с проектом *.vcxproj на C++.

\item Пропишем проекте в файле \textbf{.h} главной формы (у меня это \textbf{MyForm.h}) строчку \textbf{\#include "HarrixMathLibrary.h"}:
\begin{lstlisting}[label=install_code_07,caption=Подключение библиотеки]
#pragma once
#include "HarrixMathLibrary.h"
...
\end{lstlisting}

\item Добавим в проект файлы \textbf{HarrixMathLibrary.cpp} и \textbf{mtrand.cpp} через правый клик по проекту: \textbf{Добавить} $\rightarrow$ \textbf{Существующий элемент Shift+Alt+A}.

\item Если планируем использовать функции, использующие случайные числа (если не знаем, то тоже сделаем), то в конструкторе главной формы инициализируем генератор случайных чисел:
\begin{lstlisting}[label=install_code_08,caption=Инициализация генератора случайных чисел]
	public ref class MyForm : public System::Windows::Forms::Form
	{
	public:
		MyForm(void)
		{
			MHL_SeedRandom();//Инициализировали генератор случайных чисел
			InitializeComponent();
			//
			//TODO: добавьте код конструктора
			//
		}
...
\end{lstlisting}

\item Теперь библиотека готова к работе, и можем ее использовать. Например, создадим кнопку button1 и listBox1 и в клике на button1 пропишем:
\begin{lstlisting}[label=install_code_09,caption=Пример использования]
private: System::Void button1_Click(System::Object^  sender, System::EventArgs^  e) {
 		double x=MHL_RandomNumber();//получим случайное число
 		listBox1->Items->Add("x = " + x.ToString());//выведем его
 		 }
\end{lstlisting}
\end{itemize}

Как видите, алгоритм подключения почти одинаков.