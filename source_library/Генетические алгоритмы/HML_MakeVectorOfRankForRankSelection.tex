\textbf{Входные параметры:}

Fitness --- массив пригодностей (можно подавать не массив пригодностей, а массив значений целевой функции, но только для задач безусловной оптимизации);
 
VHML\_ResultVector --- массив рангов, который мы и формируем;
 
VHML\_N --- размер массива пригодностей.

\textbf{Возвращаемое значение:} 
 
Отсутствует.
 
\textbf{О функции:}

Это служебная функция для использования функции ранговой селекции HML\_RankSelection. Формирование вектора происходит согласно правилам ранговой селекции из ГА. Проставляет ранги для элементов несортированного массива, то есть номера, начиная с 1, в отсортированном массиве. Если в массиве есть несколько одинаковых элементов, то ранги им присуждаются как среднеарифметические.

Работает в связке с функциями HML\_RankSelection и HML\_MakeVectorOfProbabilityForProportionalSelectionV2. Оператор селекции работает с массивом пригодностей индивидов, но непосредственно ранговая селекция выбирает индивида исходя из рангов индивидов, преобразованных в вероятности выбора. Каждый раз для выбора индивида создавать массив вероятностей и рангов затратно, поэтому для каждой популяции на каждом поколении вначале вызывается функция HML\_MakeVectorOfRankForRankSelection для генерации вектора рангов, а затем HML\_MakeVectorOfProbabilityForProportionalSelectionV2 для генерации вектора вероятностей выбора индивида, а затем этот массив и подставляется в ранговую селекцию.

\textbf{Примечание:}

 Под массивом пригодностей понимается специально преобразованный массив значений целевой функции. Процесс подробно описан в стандарте генетического алгоритма. Смотреть здесь. Но это если Вы используете в алгоритмах оптимизации подобных генетическому. а так, если будете использовать, то учитывайте, что массив пригодностей --- это массив вещественных чисел из отрезка $[0;1]$.
