\textbf{Входные параметры:}
 
Parent1 --- первый родитель;
 
Parent2 --- второй родитель;
 
VMHL\_ResultVector --- потомок;

w --- параметр скрещивания, который означает долю второго родителя в потомке: $[-0.25;1.25]$;
 
VMHL\_N --- размер векторов Parent1, Parent2 и VMHL\_ResultVector.

\textbf{Возвращаемое значение:}

 Отсутствует.
 
\textbf{ Примечание:}

 Потомок только один.
 
Данный оператор скрещивания используется для вещественных векторов.

Пусть имеется два родителя (родительские хромосомы) $ \overline{Parent}^1 $ и $ \overline{Parent}^2$ длины $n$. Гены потомка определяются как некоторая фиксированная доля генов первого и второго родителя То есть скрещивание происходит по формулам:
\begin{align}
\label{SetOfOperatorsAlgorithms:eq:ExtendedLineForReal}
&Crossover \left( \overline{Parent}^1, \overline{Parent}^2, DataOfCros\right)=\overline{Offspring}, \\
& \overline{Offspring}_i=\overline{Parent}^1_i+w\cdot\overline{Parent}^2_i , i=\overline{1,n};\nonumber\\
&\overline{Offspring}^1\in X, w\in \left[ -0.25; 1.25\right] .\nonumber
\end{align}

Расширенное линейчатое скрещивание для вещественных векторов с возвращением добавляет в $ DataOfCros $ дополнительный параметр --- параметр скрещивания, который означает долю второго родителя в потомке $ w $. Обычно выбирают значение этого параметра равное $ w=0.5 $.

\begin{equation}
DataOfCros=\left( \begin{array}{c} TypeOfCros \\ w \end{array} \right).
\end{equation}