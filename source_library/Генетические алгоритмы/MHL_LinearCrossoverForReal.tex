\textbf{Входные параметры:}
 
Parent1 --- первый родитель;
 
Parent2 --- второй родитель;
 
VMHL\_ResultVector --- потомок;
 
VMHL\_N --- размер векторов Parent1, Parent2 и VMHL\_ResultVector.

\textbf{Возвращаемое значение:}

 Отсутствует.
 
\textbf{ Примечание:}

 Потомок выбирается случайно.
 
Данный оператор скрещивания используется для вещественных векторов.

Пусть имеется два родителя (родительские хромосомы) $ \overline{Parent}^1 $ и $ \overline{Parent}^2$ длины $n$. Формируется три потомка: один как среднеарифметическое генов родителей, второй и третий аналогичным способом, но с неравномерными долями.  Из них выбирается случайно один потомок, который и передается в качестве результата оператора скрещивания. То есть скрещивание происходит по формулам:
\begin{align}
\label{SetOfOperatorsAlgorithms:eq:LinearCrossoverForReal}
&Crossover \left( \overline{Parent}^1, \overline{Parent}^2, DataOfCros\right)=Random \left(\left\lbrace \overline{Offspring}^j\right\rbrace  \right), j=\overline{1,3}, \\
& \overline{Offspring}^1_i=0.5\cdot\overline{Parent}^1_i+0.5\cdot\overline{Parent}^2_i , i=\overline{1,n};\nonumber\\
&\overline{Offspring}^2_i=1.5\cdot\overline{Parent}^1_i-0.5\cdot\overline{Parent}2_i , i=\overline{1,n};\nonumber\\
&\overline{Offspring}^3_i=-0.5\cdot\overline{Parent}^1_i+1.5\cdot\overline{Parent}^2_i , i=\overline{1,n}.\nonumber
\end{align}

$ DataOfCros $ не содержит каких-либо параметров относительно данного типа скрещивания.