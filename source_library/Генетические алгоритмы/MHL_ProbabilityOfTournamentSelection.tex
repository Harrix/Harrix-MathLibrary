\textbf{Входные параметры:}  
 
 Fitness --- указатель на вектор значений целевой функции (не пригодности) индивидов;
 
 VMHL\_ResultVector\_Probability --- указатель на вектор, в который будет проводиться запись;
 
 T --- размер турнира;
 
 VMHL\_N ---  размер массивов.

\textbf{Возвращаемое значение:}

 Сумму вектора вероятностей Probability.
 
  \textbf{Примечание:}
  
   Данная функция не нужна для работы турнирной селекции через функцию MHL\_TournamentSelection в генетическом алгоритме. Функция предназначена для научных изысканий по исследованию работы  различных видов селекций.
 
 \textbf{Формула:}
 
\begin{equation*}
P\left( X_j\right) = \dfrac{\sum_{j=\max \left(1, T-\left( N-n_1-n_0\right)  \right) }^{\min \left( T, n_1\right) }C_{n_1}^j\cdot C_{ N-n_1-n_0}^{T-j}}{n_1\cdot C_N^T}, \text{ где}
\end{equation*}
\begin{equation*}
n_0=\sum_{j=1}^{N} S_0\left( X_j\right), S_0\left( X_j\right)=\left\lbrace \begin{aligned} 1, \text{ если } f\left( X_j\right)> f\left( X_i\right); \\ 0, \text{ если } f\left( X_j\right)\leq f\left( X_i\right). \end{aligned}\right.
\end{equation*}
\begin{equation*}
n_1=\sum_{j=1}^{N} S_1\left( X_j\right), S_1\left( X_j\right)=\left\lbrace \begin{aligned} 1, \text{ если } f\left( X_j\right)= f\left( X_i\right); \\ 0, \text{ если } f\left( X_j\right)\neq f\left( X_i\right). \end{aligned}\right.
\end{equation*}