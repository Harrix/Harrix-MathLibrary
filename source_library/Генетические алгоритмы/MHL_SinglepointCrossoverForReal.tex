\textbf{Входные параметры:}
 
Parent1 --- первый родитель;
 
Parent2 --- второй родитель;
 
VMHL\_ResultVector --- потомок;
 
VMHL\_N --- размер векторов Parent1, Parent2 и VMHL\_ResultVector.

\textbf{Возвращаемое значение:}

 Отсутствует.
 
\textbf{ Примечание:}

 Потомок выбирается случайно.
 
По сути ничем не отличается от \hyperref[SetOfOperatorsAlgorithms:SinglepointCrossover]{SinglepointCrossover}, кроме типа векторов, на котором работает оператор.
 
 \textbf{Формула:}
\begin{align}
&Crossover \left( \overline{Parent}^1, \overline{Parent}^2, DataOfCros\right)=Random \left(\left\lbrace \overline{Offspring}^1; \overline{Offspring}^2\right\rbrace  \right), \nonumber\\
&R=Random\left( \left\lbrace 2; 3; \ldots; n\right\rbrace \right); \nonumber \\
& \overline{Offspring}^1_i=\overline{Parent}^1_i, i=\overline{1,R-1};\nonumber\\
&  \overline{Offspring}^1_i=\overline{Parent}^2_i, i=\overline{R,n};\nonumber\\
&\overline{Offspring}^2_i=\overline{Parent}^2_i, i=\overline{1,R-1};\nonumber\\
& \overline{Offspring}^2_i=\overline{Parent}^1_i, i=\overline{R,n};\nonumber\\
&\overline{Offspring}^1\in X, \overline{Offspring}^2\in X.\nonumber
\end{align}

$ DataOfCros $ не содержит каких-либо параметров относительно данного типа скрещивания.