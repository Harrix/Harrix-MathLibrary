\textbf{Входные параметры:}

 Fitness --- массив пригодностей (можно подавать не массив пригодностей, а массив значений целевой функции, но только для задач безусловной оптимизации);
 
 VMHL\_ResultVector --- массив рангов, который мы и формируем;
 
 VMHL\_N --- размер массива пригодностей.
 
\textbf{Возвращаемое значение:} 
 
Отсутствует.
 
\textbf{О функции:}

Это модифицированная функция. Оригинальная функция MHL\_MakeVectorOfRankForRankSelectionпроставляет ранги с 1.

\textbf{Примечание:}

 Под массивом пригодностей понимается специально преобразованный массив значений целевой функции. Процесс подробно описан в стандарте генетического алгоритма. Смотреть здесь. Но это если Вы используете в алгоритмах оптимизации подобных генетическому. а так, если будете использовать, то учитывайте, что массив пригодностей --- это массив вещественных чисел из отрезка $[0;1]$.
