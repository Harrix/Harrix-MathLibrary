\textbf{Входные параметры:}
 
Parent1 --- первый родитель;
 
Parent2 --- второй родитель;
 
VHML\_ResultVector --- потомок;

alpha --- параметр скрещивания, который означает своеобразную долю какого-то родителя в потомке: $[0;1]$;
 
VHML\_N --- размер векторов Parent1, Parent2 и VHML\_ResultVector.

\textbf{Возвращаемое значение:}

 Отсутствует.
 
\textbf{ Примечание:}

Потомок только один.
 
Данный оператор скрещивания используется для вещественных векторов.

Пусть имеется два родителя (родительские хромосомы) $ \overline{Parent}^1 $ и $ \overline{Parent}^2$ длины $n$. Данное скрещивание похоже на плоское скрещивание (\hyperref[SetOfOperatorsAlgorithms:FlatCrossoverForReal]{FlatCrossoverForReal}), то есть гены потомка выбираются как случайное число в границах,обозначенных генами родителей. Но в данном скрещивании область увеличена. То есть скрещивание происходит по формулам:
\begin{align}
\label{SetOfOperatorsAlgorithms:eq:BLXCrossoverForReal}
&Crossover \left( \overline{Parent}^1, \overline{Parent}^2, DataOfCros\right)= \overline{Offspring}, \\
& \overline{Offspring}_i=random\left(cmin_i-I_i\cdot \alpha, cmax_i+I_i\cdot \alpha\right);\nonumber\\
& cmin_i = \min\left(\overline{Parent}^1_i, \overline{Parent}^2_i \right);\nonumber\\
& cmax_i = \max\left(\overline{Parent}^1_i, \overline{Parent}^2_i \right);\nonumber\\
& I_i = cmax_i-cmin_i;\nonumber\\
&\alpha\in \left[ 0; 1\right].\nonumber
\end{align}

BLX скрещивание для вещественных векторов с возвращением добавляет в $ DataOfCros $ дополнительный параметр --- параметр скрещивания $ \alpha $. Обычно выбирают значение этого параметра равное $ w=0.5 $.

\begin{equation}
DataOfCros=\left( \begin{array}{c} TypeOfCros \\ \alpha \end{array} \right).
\end{equation}